\documentclass[a4paper,12pt]{article}

\usepackage[utf8]{inputenc} 
\usepackage[T2A]{fontenc}  
\usepackage[ukrainian]{babel} 
\usepackage[a4paper, left=20mm, right=20mm, top=20mm, bottom=20mm]{geometry}
\usepackage{hyperref}
\usepackage{adjustbox}
\usepackage{float}
\usepackage[labelfont=bf,labelsep=emdash]{caption}
\addto\captionsukrainian{
  \renewcommand{\tablename}{Таблиця}
}
\usepackage{enumitem}
\newcommand{\altitem}[1]{\item[#1.]}
 \usepackage{amsmath} 




\begin{document}


\begin{titlepage}

\begin{center}
{\large НАЦІОНАЛЬНИЙ ТЕХНІЧНИЙ УНІВЕРСИТЕТ УКРАЇНИ} \par
{\large <<КИЇВСЬКИЙ ПОЛІТЕХНІЧНИЙ ІНСТИТУТ ім. Ігоря СІКОРСЬКОГО>>}\par
{\large НАВЧАЛЬНО-НАУКОВИЙ ФІЗИКО-ТЕХНІЧНИЙ ІНСТИТУТ}\par

\vspace{60mm}
{\huge Звіт з комп'ютерного практикуму 1 \par}
\vspace{5mm}
{\Large Варіант 4 \par}
\end{center}

\vspace{60mm}
\begin{flushright}
{\Large Виконали студенти}

{\Large групи ФІ-52мн}

{\Large Ісаченко Нікіта, Фурутіна Євгенія}

\vspace{30mm}
\end{flushright}
\vspace{20mm}

\begin{center}
{\Large 2025}
\end{center}



\end{titlepage}

\section{Мета роботи.}
Ознайомлення з основами баєсiвського пiдходу в криптоаналiзi, побудова детермiнiстичної та стохастичної вирішуючих функцiй для моделей схем шифрування, а також криптоаналiз моделей шифрiв за допомогою програмної реалiзацiї з проведенням порiвняльного аналiзу вирішуючих функцiй.


\section{Постановка задачі.}
Для заданого (4-го) варiанта моделi шифру описати та програмно реалізувати алгоритми побудови оптимальних детермiнiстичної та стохастичної вирiшуючих функцiй; обчислити середні втрати та провести порівняльний аналіз вирішуючих функцій.

\section{Хід роботи.}
\begin{enumerate}
  \item Створити репозиторій. 
  
  \url{https://github.com/misach2135/methods-of-cryptoanalysis-lab-1}
  
  \item Реалiзувати алгоритми для побудови детермiнiстичної та стохастичної вирiшуючих функцiй і подати результати у вигляді таблиць. Для реалізації функцій потрібно: 
     \begin{enumerate}
        \item порахувати розподiли P(C) та P(M, C);
        \item ґрунтуючись на цих розподiлах, обчислити P(M|C);
        \item побудувати оптимальні детермiнiстичну та стохастичну вирiшуючі функцiї, використовуючи максимiзацiю P(M|C).
     \end{enumerate}
  \item Обчислити середні втрати для обох функцій.
  \item Провести порiвняльний аналiз вирiшуючих функцiй.
  
\end{enumerate}


\section{Вхідні дані.}
Маємо модель криптосистеми, для якої $|M| = |K| = |C| = 20$, а також дві таблиці: з ймовірнисним розподілом для $M$ і $K$ (Таблиця \ref{Tab1}) та  з індексами шифротекстів, що відповідають деяким $K_i$ і $M_j$ (Таблиця \ref{Tab2}). 

\vspace{2em}

\begin{table}[H]
    \centering
    \begin{adjustbox}{scale=0.8,center}
    \begin{tabular}{|*{21}{c|}}
    \hline
        $\mathbf{P(M)}$ & 0.09 & 0.09 & 0.09 & 0.09 & 0.04 & 0.04 & 0.04 & 0.04 & 0.04 & 0.04 & 0.04 & 0.04 & 0.04 & 0.04 & 0.04 & 0.04 & 0.04 & 0.04 & 0.04 & 0.04 \\ \hline
        $\mathbf{P(k)}$ & 0.05 & 0.05 & 0.05 & 0.05 & 0.05 & 0.05 & 0.05 & 0.05 & 0.05 & 0.05 & 0.05 & 0.05 & 0.05 & 0.05 & 0.05 & 0.05 & 0.05 & 0.05 & 0.05 & 0.05 \\ \hline
    \end{tabular}
    \end{adjustbox}
    \caption{}
    \label{Tab1}
\end{table}

\begin{table}[H]
    \centering
    \begin{adjustbox}{scale=0.82,center}
    \begin{tabular}{|*{21}{c|}}
    \hline
        & $\mathbf{M_0}$ & $\mathbf{M_1}$ & $\mathbf{M_2}$ & $\mathbf{M_3}$ & $\mathbf{M_4}$ & $\mathbf{M_5}$ & $\mathbf{M_6}$ & $\mathbf{M_7}$ & $\mathbf{M_8}$ & $\mathbf{M_9}$ & $\mathbf{M_{10}}$ & $\mathbf{M_{11}}$ & $\mathbf{M_{12}}$ & $\mathbf{M_{13}}$ & $\mathbf{M_{14}}$ & $\mathbf{M_{15}}$ & $\mathbf{M_{16}}$ & $\mathbf{M_{17}}$ & $\mathbf{M_{18}}$ & $\mathbf{M_{19}}$ \\
    \hline
        $\mathbf{k_0}$ & 3 & 11 & 17 & 19 & 12 & 6 & 2 & 15 & 13 & 14 & 1 & 18 & 5 & 0 & 16 & 8 & 10 & 7 & 4 & 9 \\ \hline
        $\mathbf{k_1}$ & 17 & 5 & 6 & 15 & 12 & 10 & 8 & 16 & 4 & 0 & 7 & 2 & 3 & 18 & 11 & 1 & 13 & 19 & 9 & 14 \\ \hline
        $\mathbf{k_2}$ & 13 & 15 & 8 & 6 & 7 & 11 & 18 & 2 & 10 & 14 & 17 & 4 & 1 & 19 & 12 & 16 & 3 & 0 & 5 & 9 \\ \hline
        $\mathbf{k_3}$ & 15 & 7 & 10 & 19 & 5 & 0 & 14 & 3 & 11 & 6 & 12 & 9 & 17 & 18 & 2 & 13 & 4 & 16 & 1 & 8 \\ \hline
        $\mathbf{k_4}$ & 9 & 4 & 1 & 7 & 6 & 2 & 14 & 0 & 13 & 8 & 5 & 3 & 19 & 18 & 11 & 12 & 10 & 16 & 15 & 17 \\ \hline
        $\mathbf{k_5}$ & 17 & 2 & 16 & 7 & 15 & 6 & 1 & 18 & 10 & 3 & 9 & 4 & 19 & 5 & 11 & 12 & 14 & 8 & 13 & 0 \\ \hline
        $\mathbf{k_6}$ & 13 & 10 & 19 & 4 & 18 & 2 & 7 & 9 & 15 & 16 & 3 & 1 & 17 & 11 & 6 & 12 & 8 & 0 & 5 & 14 \\ \hline
        $\mathbf{k_7}$ & 11 & 4 & 1 & 16 & 9 & 13 & 19 & 8 & 18 & 5 & 15 & 10 & 3 & 7 & 0 & 17 & 14 & 2 & 6 & 12 \\ \hline
        $\mathbf{k_8}$ & 16 & 19 & 6 & 10 & 0 & 15 & 7 & 13 & 4 & 2 & 14 & 3 & 17 & 5 & 18 & 11 & 1 & 9 & 8 & 12 \\ \hline
        $\mathbf{k_9}$ & 1 & 12 & 14 & 15 & 2 & 19 & 0 & 3 & 7 & 6 & 8 & 18 & 4 & 11 & 10 & 5 & 9 & 13 & 16 & 17 \\ \hline
        $\mathbf{k_{10}}$ & 8 & 2 & 6 & 5 & 18 & 16 & 9 & 1 & 19 & 17 & 15 & 7 & 0 & 10 & 4 & 3 & 14 & 13 & 12 & 11 \\ \hline
        $\mathbf{k_{11}}$ & 11 & 2 & 13 & 7 & 18 & 8 & 9 & 1 & 14 & 12 & 4 & 17 & 10 & 15 & 16 & 6 & 0 & 3 & 19 & 5 \\ \hline
        $\mathbf{k_{12}}$ & 16 & 3 & 6 & 18 & 11 & 8 & 17 & 15 & 13 & 10 & 5 & 2 & 1 & 7 & 12 & 19 & 4 & 9 & 0 & 14 \\ \hline
        $\mathbf{k_{13}}$ & 12 & 4 & 0 & 8 & 13 & 10 & 9 & 6 & 17 & 3 & 1 & 14 & 16 & 15 & 5 & 19 & 18 & 11 & 7 & 2 \\ \hline
        $\mathbf{k_{14}}$ & 12 & 10 & 17 & 2 & 6 & 16 & 3 & 15 & 9 & 13 & 4 & 1 & 19 & 14 & 11 & 0 & 8 & 18 & 7 & 5 \\ \hline
        $\mathbf{k_{15}}$ & 19 & 16 & 10 & 7 & 8 & 14 & 15 & 1 & 6 & 13 & 5 & 12 & 4 & 11 & 17 & 3 & 18 & 0 & 2 & 9 \\ \hline
        $\mathbf{k_{16}}$ & 12 & 4 & 13 & 17 & 10 & 9 & 0 & 15 & 2 & 19 & 16 & 1 & 7 & 5 & 11 & 8 & 6 & 14 & 3 & 18 \\ \hline
        $\mathbf{k_{17}}$ & 15 & 8 & 3 & 12 & 1 & 14 & 10 & 11 & 17 & 16 & 9 & 4 & 5 & 6 & 0 & 7 & 19 & 13 & 2 & 18 \\ \hline
        $\mathbf{k_{18}}$ & 11 & 3 & 12 & 7 & 1 & 2 & 9 & 14 & 4 & 17 & 6 & 19 & 8 & 10 & 16 & 18 & 0 & 5 & 13 & 15 \\ \hline
        $\mathbf{k_{19}}$ & 2 & 9 & 18 & 15 & 0 & 6 & 16 & 13 & 12 & 4 & 19 & 10 & 11 & 1 & 7 & 17 & 3 & 8 & 5 & 14 \\ \hline
    \end{tabular}
\end{adjustbox}
    \caption{}
    \label{Tab2}
\end{table}

\vspace{1em}


\section{Опис алгоритму побудови оптимальної детермiнiстичної вирішуючої функції.}

\begin{enumerate}
  \item Знаходимо $P(M|C) = \frac{P(M,C)}{P(C)}$, використовуючи наступні формули:
  \begin{itemize}
  \item $P(C) = \sum_{\forall (M,k): E_{k}(M) = C}{P(M,k)}$ -- враховуємо всі пари $(M,k)$, які після шифрування дають саме цей $C$ (Таблиця \ref{Tab6});
  \item $P(M,C) = \sum_{\forall k: E_{k}(M) = C}{P(M,k)}$ -- дивимося на всі ключі, які перетворюються саме цей $M$ у саме цей $C$ (Таблиця \ref{Tab7});
  \item $P(M,k) = P(M) \cdot P(k)$ -- рахуємо за допомогою Таблиці \ref{Tab1}.
  \end{itemize}
  \item З таблиці усіх значень $P(M|C)$ (Таблиця \ref{Tab3}) обираємо найбільше значення для кожного $C$. Шукаємо відповідні відкриті тексти, з якими цей шифротекст формує цю максимальну ймовірність, і виписуємо індекси таких $M$ (для кожного $C$ -- один фіксований $M$).  Ця послідовність відкритих текстів і буде оптимальною детерміністичною вирішуючою функцією (Таблиця \ref{Tab4}). Це випливає з того, що $\delta_D$ є оптимальною тоді і тільки тоді, коли вона є баєсівською, тобто виконується умова $P(\delta_B(C)|C) = max_{M \in \mathcal{M}}P(M|C)$.
  \altitem{2*} \textbf{Альтернативний варіант:} робити не за максимізацією, а через підрахунок середніх втрат. У цьому випадку розглядаємо усі можливі детерміністичні вирішуючі функції і рахуємо для кожної з них середню втрату за формулою $$l_{\delta_D} = \sum_{M \in \mathcal{M}} \sum_{C \in \mathcal{C}}{P(M,C) \cdot L_{\delta_D}(M,C)},$$ де 
\begin{equation*}
 L_{\delta_D}(M,C) =      \left\{
                           \begin{array}{l}
                            1, \delta_D(C) \neq M \\
                            0, \delta_D(C) = M 
                            \end{array}
                          \right. \quad \text{— функція втрат.} \end{equation*} 
Після цього шукаємо найменше число серед усіх значень середніх втрат і дивимося, яка функція цій величині відповідає. Це і є оптимальна детерміністична вирішуюча функція. 
\end{enumerate}

\section{Опис алгоритму побудови оптимальної стохастичної вирішуючої функції.}

\begin{enumerate}

\item Формуємо таблицю зі значеннями $P(M|C)$ таким же чином, як і для детерміністичної вирішуючої функції (Таблиця \ref{Tab3}).
\item Шукаємо максимуми з цих значень для кожного $C$.  
\item На відміну від випадку з детерміністичною вирішуючою функцією, де ми шукали тільки один $M$ для кожного $C$, у стохастичній вирішуючій функції шукаємо усі такі $M$, за яких значення $P(M|C)$ досягається максимуму.  
\item Рахуємо кількість таких $M$ для кожного $C$, після чого формуємо нову таблицю $20 \times 20$, в кожній клітинці якої стоїть значення $\frac{1}{k}$, де $k$ -- кількість таких відкритів текстів для заданого шифротексту (Таблиця \ref{Tab5}).  


\end{enumerate}

\section{Таблиця ймовірностей P(M|C).}

\begin{table}[H]
    \centering
    \begin{adjustbox}{scale=0.63,center}
    \begin{tabular}{|*{21}{c|}}
        \hline
        $\mathbf{P(M|C)}$ & $\mathbf{C_0}$ & $\mathbf{C_1}$ & $\mathbf{C_2}$ & $\mathbf{C_3}$ & $\mathbf{C_4}$ & $\mathbf{C_5}$ & $\mathbf{C_6}$ & $\mathbf{C_7}$ & $\mathbf{C_8}$ & $\mathbf{C_9}$ & $\mathbf{C_{10}}$ & $\mathbf{C_{11}}$ & $\mathbf{C_{12}}$ & $\mathbf{C_{13}}$ & $\mathbf{C_{14}}$ & $\mathbf{C_{15}}$ & $\mathbf{C_{16}}$ & $\mathbf{C_{17}}$ & $\mathbf{C_{18}}$ & $\mathbf{C_{19}}$ \\
    \hline
        $\mathbf{M_0}$ & 0.0000 & 0.0947 & 0.0857 & 0.0900 & 0.0000 & 0.0000 & 0.0000 & 0.0000 & 0.0900 & 0.1000 & 0.0000 & 0.2700 & 0.2455 & 0.1800 & 0.0000 & 0.1636 & 0.1714 & 0.1714 & 0.0000 & 0.0857 \\ \hline
        $\mathbf{M_1}$ & 0.0000 & 0.0000 & 0.2571 & 0.1800 & 0.3429 & 0.1000 & 0.0000 & 0.0818 & 0.0900 & 0.1000 & 0.1714 & 0.0900 & 0.0818 & 0.0000 & 0.0000 & 0.0818 & 0.0857 & 0.0000 & 0.0000 & 0.0857 \\ \hline
        $\mathbf{M_2}$ & 0.1059 & 0.1895 & 0.0000 & 0.0900 & 0.0000 & 0.0000 & 0.3429 & 0.0000 & 0.0900 & 0.0000 & 0.1714 & 0.0000 & 0.0818 & 0.1800 & 0.1059 & 0.0000 & 0.0857 & 0.1714 & 0.1000 & 0.0857 \\ \hline
        $\mathbf{M_3}$ & 0.0000 & 0.0000 & 0.0857 & 0.0000 & 0.0857 & 0.1000 & 0.0857 & 0.4091 & 0.0900 & 0.0000 & 0.0857 & 0.0000 & 0.0818 & 0.0000 & 0.0000 & 0.2455 & 0.0857 & 0.0857 & 0.1000 & 0.1714 \\ \hline
        $\mathbf{M_4}$ & 0.0941 & 0.0842 & 0.0381 & 0.0000 & 0.0000 & 0.0444 & 0.0762 & 0.0364 & 0.0400 & 0.0444 & 0.0381 & 0.0400 & 0.0727 & 0.0400 & 0.0000 & 0.0364 & 0.0000 & 0.0000 & 0.1333 & 0.0000 \\ \hline
        $\mathbf{M_5}$ & 0.0471 & 0.0000 & 0.1143 & 0.0000 & 0.0000 & 0.0000 & 0.1143 & 0.0000 & 0.0800 & 0.0444 & 0.0762 & 0.0400 & 0.0000 & 0.0400 & 0.0941 & 0.0364 & 0.0762 & 0.0000 & 0.0000 & 0.0381 \\ \hline
        $\mathbf{M_6}$ & 0.0941 & 0.0421 & 0.0381 & 0.0400 & 0.0000 & 0.0000 & 0.0000 & 0.0727 & 0.0400 & 0.1778 & 0.0381 & 0.0000 & 0.0000 & 0.0000 & 0.0941 & 0.0364 & 0.0381 & 0.0381 & 0.0444 & 0.0381 \\ \hline
        $\mathbf{M_7}$ & 0.0471 & 0.1263 & 0.0381 & 0.0800 & 0.0000 & 0.0000 & 0.0381 & 0.0000 & 0.0400 & 0.0444 & 0.0000 & 0.0400 & 0.0000 & 0.0800 & 0.0471 & 0.1455 & 0.0381 & 0.0000 & 0.0444 & 0.0000 \\ \hline
        $\mathbf{M_8}$ & 0.0000 & 0.0000 & 0.0381 & 0.0000 & 0.1143 & 0.0000 & 0.0381 & 0.0364 & 0.0000 & 0.0444 & 0.0762 & 0.0400 & 0.0364 & 0.1200 & 0.0471 & 0.0364 & 0.0000 & 0.0762 & 0.0444 & 0.0381 \\ \hline
        $\mathbf{M_9}$ & 0.0471 & 0.0000 & 0.0381 & 0.0800 & 0.0381 & 0.0444 & 0.0762 & 0.0000 & 0.0400 & 0.0000 & 0.0381 & 0.0000 & 0.0364 & 0.0800 & 0.0941 & 0.0000 & 0.0762 & 0.0762 & 0.0000 & 0.0381 \\ \hline
        $\mathbf{M_{10}}$ & 0.0000 & 0.0842 & 0.0000 & 0.0400 & 0.0762 & 0.1333 & 0.0381 & 0.0364 & 0.0400 & 0.0889 & 0.0000 & 0.0000 & 0.0364 & 0.0000 & 0.0471 & 0.0727 & 0.0381 & 0.0381 & 0.0000 & 0.0381 \\ \hline
        $\mathbf{M_{11}}$ & 0.0000 & 0.1263 & 0.0762 & 0.0800 & 0.1143 & 0.0000 & 0.0000 & 0.0364 & 0.0000 & 0.0444 & 0.0762 & 0.0000 & 0.0364 & 0.0000 & 0.0471 & 0.0000 & 0.0000 & 0.0381 & 0.0889 & 0.0381 \\ \hline
        $\mathbf{M_{12}}$ & 0.0471 & 0.0842 & 0.0000 & 0.0800 & 0.0762 & 0.0889 & 0.0000 & 0.0364 & 0.0400 & 0.0000 & 0.0381 & 0.0400 & 0.0000 & 0.0000 & 0.0000 & 0.0000 & 0.0381 & 0.1143 & 0.0000 & 0.1143 \\ \hline
        $\mathbf{M_{13}}$ & 0.0471 & 0.0421 & 0.0000 & 0.0000 & 0.0000 & 0.1333 & 0.0381 & 0.0727 & 0.0000 & 0.0000 & 0.0762 & 0.1200 & 0.0000 & 0.0000 & 0.0471 & 0.0727 & 0.0000 & 0.0000 & 0.1333 & 0.0381 \\ \hline
        $\mathbf{M_{14}}$ & 0.0941 & 0.0000 & 0.0381 & 0.0000 & 0.0381 & 0.0444 & 0.0381 & 0.0364 & 0.0000 & 0.0000 & 0.0381 & 0.2000 & 0.0727 & 0.0000 & 0.0000 & 0.0000 & 0.1143 & 0.0381 & 0.0444 & 0.0000 \\ \hline
        $\mathbf{M_{15}}$ & 0.0471 & 0.0421 & 0.0000 & 0.0800 & 0.0000 & 0.0444 & 0.0381 & 0.0364 & 0.0800 & 0.0000 & 0.0000 & 0.0400 & 0.1091 & 0.0400 & 0.0000 & 0.0000 & 0.0381 & 0.0762 & 0.0444 & 0.0762 \\ \hline
        $\mathbf{M_{16}}$ & 0.0941 & 0.0421 & 0.0000 & 0.0800 & 0.0762 & 0.0000 & 0.0381 & 0.0000 & 0.0800 & 0.0444 & 0.0762 & 0.0000 & 0.0000 & 0.0400 & 0.1412 & 0.0000 & 0.0000 & 0.0000 & 0.0889 & 0.0381 \\ \hline
        $\mathbf{M_{17}}$ & 0.1412 & 0.0000 & 0.0381 & 0.0400 & 0.0000 & 0.0444 & 0.0000 & 0.0364 & 0.0800 & 0.0889 & 0.0000 & 0.0400 & 0.0000 & 0.1200 & 0.0471 & 0.0000 & 0.0762 & 0.0000 & 0.0444 & 0.0381 \\ \hline
        $\mathbf{M_{18}}$ & 0.0471 & 0.0421 & 0.0762 & 0.0400 & 0.0381 & 0.1333 & 0.0381 & 0.0727 & 0.0400 & 0.0444 & 0.0000 & 0.0000 & 0.0364 & 0.0800 & 0.0000 & 0.0364 & 0.0381 & 0.0000 & 0.0000 & 0.0381 \\ \hline
        $\mathbf{M_{19}}$ & 0.0471 & 0.0000 & 0.0381 & 0.0000 & 0.0000 & 0.0889 & 0.0000 & 0.0000 & 0.0400 & 0.1333 & 0.0000 & 0.0400 & 0.0727 & 0.0000 & 0.1882 & 0.0364 & 0.0000 & 0.0762 & 0.0889 & 0.0000 \\ \hline
    \end{tabular}
    \end{adjustbox}
    \caption{}
    \label{Tab3}
\end{table}

\section{Знайдена оптимальна детерміністична вирішуюча функція.}

\begin{table}[H]
    \centering
    \begin{adjustbox}{scale=1.4,center}
    \begin{tabular}{|*{21}{c|}}
    \hline
        $\mathbf{\delta_D}$ & 17 & 2 & 1 & 1 & 1 & 18 & 2 & 3 & 3 & 6 & 2 & 0 & 0 & 2 & 19 & 3 & 0 & 2 & 13 & 3 \\ \hline
    \end{tabular}
\end{adjustbox}
    \caption{}
    \label{Tab4}
\end{table}

\section{Знайдена оптимальна стохастична вирішуюча функція.}

\begin{table}[H]
    \centering
    \begin{adjustbox}{scale=0.65,center}
    \begin{tabular}{|*{21}{c|}}
      \hline
        $\mathbf{\delta_S}$ & $\mathbf{C_0}$ & $\mathbf{C_1}$ & $\mathbf{C_2}$ & $\mathbf{C_3}$ & $\mathbf{C_4}$ & $\mathbf{C_5}$ & $\mathbf{C_6}$ & $\mathbf{C_7}$ & $\mathbf{C_8}$ & $\mathbf{C_9}$ & $\mathbf{C_{10}}$ & $\mathbf{C_{11}}$ & $\mathbf{C_{12}}$ & $\mathbf{C_{13}}$ & $\mathbf{C_{14}}$ & $\mathbf{C_{15}}$ & $\mathbf{C_{16}}$ & $\mathbf{C_{17}}$ & $\mathbf{C_{18}}$ & $\mathbf{C_{19}}$ \\
    \hline
        $\mathbf{M_0}$ & 0.0000 & 0.0000 & 0.0000 & 0.0000 & 0.0000 & 0.0000 & 0.0000 & 0.0000 & 0.0000 & 0.0000 & 0.0000 & 0.0000 & 0.0000 & 0.0000 & 0.0000 & 0.0000 & 0.0000 & 1.0000 & 0.0000 & 0.0000 \\ \hline
        $\mathbf{M_1}$ & 0.0000 & 0.0000 & 1.0000 & 0.0000 & 0.0000 & 0.0000 & 0.0000 & 0.0000 & 0.0000 & 0.0000 & 0.0000 & 0.0000 & 0.0000 & 0.0000 & 0.0000 & 0.0000 & 0.0000 & 0.0000 & 0.0000 & 0.0000 \\ \hline
        $\mathbf{M_2}$ & 0.0000 & 1.0000 & 0.0000 & 0.0000 & 0.0000 & 0.0000 & 0.0000 & 0.0000 & 0.0000 & 0.0000 & 0.0000 & 0.0000 & 0.0000 & 0.0000 & 0.0000 & 0.0000 & 0.0000 & 0.0000 & 0.0000 & 0.0000 \\ \hline
        $\mathbf{M_3}$ & 0.0000 & 1.0000 & 0.0000 & 0.0000 & 0.0000 & 0.0000 & 0.0000 & 0.0000 & 0.0000 & 0.0000 & 0.0000 & 0.0000 & 0.0000 & 0.0000 & 0.0000 & 0.0000 & 0.0000 & 0.0000 & 0.0000 & 0.0000 \\ \hline
        $\mathbf{M_4}$ & 0.0000 & 1.0000 & 0.0000 & 0.0000 & 0.0000 & 0.0000 & 0.0000 & 0.0000 & 0.0000 & 0.0000 & 0.0000 & 0.0000 & 0.0000 & 0.0000 & 0.0000 & 0.0000 & 0.0000 & 0.0000 & 0.0000 & 0.0000 \\ \hline
        $\mathbf{M_5}$ & 0.0000 & 0.0000 & 0.0000 & 0.0000 & 0.0000 & 0.0000 & 0.0000 & 0.0000 & 0.0000 & 0.0000 & 0.3333 & 0.0000 & 0.0000 & 0.3333 & 0.0000 & 0.0000 & 0.0000 & 0.0000 & 0.3333 & 0.0000 \\ \hline
        $\mathbf{M_6}$ & 0.0000 & 0.0000 & 1.0000 & 0.0000 & 0.0000 & 0.0000 & 0.0000 & 0.0000 & 0.0000 & 0.0000 & 0.0000 & 0.0000 & 0.0000 & 0.0000 & 0.0000 & 0.0000 & 0.0000 & 0.0000 & 0.0000 & 0.0000 \\ \hline
        $\mathbf{M_7}$ & 0.0000 & 0.0000 & 0.0000 & 1.0000 & 0.0000 & 0.0000 & 0.0000 & 0.0000 & 0.0000 & 0.0000 & 0.0000 & 0.0000 & 0.0000 & 0.0000 & 0.0000 & 0.0000 & 0.0000 & 0.0000 & 0.0000 & 0.0000 \\ \hline
        $\mathbf{M_8}$ & 0.2500 & 0.2500 & 0.2500 & 0.2500 & 0.0000 & 0.0000 & 0.0000 & 0.0000 & 0.0000 & 0.0000 & 0.0000 & 0.0000 & 0.0000 & 0.0000 & 0.0000 & 0.0000 & 0.0000 & 0.0000 & 0.0000 & 0.0000 \\ \hline
        $\mathbf{M_9}$ & 0.0000 & 0.0000 & 0.0000 & 0.0000 & 0.0000 & 0.0000 & 1.0000 & 0.0000 & 0.0000 & 0.0000 & 0.0000 & 0.0000 & 0.0000 & 0.0000 & 0.0000 & 0.0000 & 0.0000 & 0.0000 & 0.0000 & 0.0000 \\ \hline
        $\mathbf{M_{10}}$ & 0.0000 & 0.5000 & 0.5000 & 0.0000 & 0.0000 & 0.0000 & 0.0000 & 0.0000 & 0.0000 & 0.0000 & 0.0000 & 0.0000 & 0.0000 & 0.0000 & 0.0000 & 0.0000 & 0.0000 & 0.0000 & 0.0000 & 0.0000 \\ \hline
        $\mathbf{M_{11}}$ & 1.0000 & 0.0000 & 0.0000 & 0.0000 & 0.0000 & 0.0000 & 0.0000 & 0.0000 & 0.0000 & 0.0000 & 0.0000 & 0.0000 & 0.0000 & 0.0000 & 0.0000 & 0.0000 & 0.0000 & 0.0000 & 0.0000 & 0.0000 \\ \hline
        $\mathbf{M_{12}}$ & 1.0000 & 0.0000 & 0.0000 & 0.0000 & 0.0000 & 0.0000 & 0.0000 & 0.0000 & 0.0000 & 0.0000 & 0.0000 & 0.0000 & 0.0000 & 0.0000 & 0.0000 & 0.0000 & 0.0000 & 0.0000 & 0.0000 & 0.0000 \\ \hline
        $\mathbf{M_{13}}$ & 0.5000 & 0.0000 & 0.5000 & 0.0000 & 0.0000 & 0.0000 & 0.0000 & 0.0000 & 0.0000 & 0.0000 & 0.0000 & 0.0000 & 0.0000 & 0.0000 & 0.0000 & 0.0000 & 0.0000 & 0.0000 & 0.0000 & 0.0000 \\ \hline
        $\mathbf{M_{14}}$ & 0.0000 & 0.0000 & 0.0000 & 0.0000 & 0.0000 & 0.0000 & 0.0000 & 0.0000 & 0.0000 & 0.0000 & 0.0000 & 0.0000 & 0.0000 & 0.0000 & 0.0000 & 0.0000 & 0.0000 & 0.0000 & 0.0000 & 1.0000 \\ \hline
        $\mathbf{M_{15}}$ & 0.0000 & 0.0000 & 0.0000 & 1.0000 & 0.0000 & 0.0000 & 0.0000 & 0.0000 & 0.0000 & 0.0000 & 0.0000 & 0.0000 & 0.0000 & 0.0000 & 0.0000 & 0.0000 & 0.0000 & 0.0000 & 0.0000 & 0.0000 \\ \hline
        $\mathbf{M_{16}}$ & 1.0000 & 0.0000 & 0.0000 & 0.0000 & 0.0000 & 0.0000 & 0.0000 & 0.0000 & 0.0000 & 0.0000 & 0.0000 & 0.0000 & 0.0000 & 0.0000 & 0.0000 & 0.0000 & 0.0000 & 0.0000 & 0.0000 & 0.0000 \\ \hline
        $\mathbf{M_{17}}$ & 0.5000 & 0.0000 & 0.5000 & 0.0000 & 0.0000 & 0.0000 & 0.0000 & 0.0000 & 0.0000 & 0.0000 & 0.0000 & 0.0000 & 0.0000 & 0.0000 & 0.0000 & 0.0000 & 0.0000 & 0.0000 & 0.0000 & 0.0000 \\ \hline
        $\mathbf{M_{18}}$ & 0.0000 & 0.0000 & 0.0000 & 0.0000 & 0.5000 & 0.0000 & 0.0000 & 0.0000 & 0.0000 & 0.0000 & 0.0000 & 0.0000 & 0.0000 & 0.5000 & 0.0000 & 0.0000 & 0.0000 & 0.0000 & 0.0000 & 0.0000 \\ \hline
        $\mathbf{M_{19}}$ & 0.0000 & 0.0000 & 0.0000 & 1.0000 & 0.0000 & 0.0000 & 0.0000 & 0.0000 & 0.0000 & 0.0000 & 0.0000 & 0.0000 & 0.0000 & 0.0000 & 0.0000 & 0.0000 & 0.0000 & 0.0000 & 0.0000 & 0.0000 \\ \hline
    \end{tabular}
    \end{adjustbox}
    \caption{}
    \label{Tab5}
\end{table}

\section{Середнi втрати для вирiшуючих функцiй.}

\begin{tabular}{ll}
$l_{\delta_D} = 0.7860000000000006$; & \\[2pt]
$l_{\delta_S} = 0.7860000000000006$. & \\
\end{tabular}

\section{Порiвняльний аналiз вирiшуючих функцiй.}

Оскільки середні втрати для обох функцій -- однакові, маємо, що вони працюють з однією й тією ж ймовірністю помилки, тобто є однаково оптимальними з огляду на результат. Оптимальна детерміністична вирішуюча функція завжди повертає фіксовану відповідь, що є найбільш ймовірною, тоді як стохастична -- обирає одну з найбільш ймовірних відповідей (їх може бути як одна, так і декілька). З погляду логіки, стохастична функція є більш `чесною`, бо рівноймовірно розподіляє вибір між декількома варіантами. Але на практиці бачимо, що переможця серед методів немає. 


\section{Опис труднощiв.}

Для реалізації програмної частини лабораторної роботи була використана мова програмування Rust, через її швидкодію, безпечний інтерфейс для роботи з пам`яттю та можливість безкоштовних (zero-cost) абстракцій. У програмній реалізації активно використовувались матриці, які були реалізовані як двовимірні масиви. Для зручності роботи з матрицями ми реалізували тип $Matrix<T, R, C>$, який містить декілька допоміжних функцій і дозволяє виводити їх у форматі csv, що є зручним для подальшої обробки (представлення в іншому форматі, парсинг тощо).

Особливих труднощів у реалізації програмної частини лабораторної роботи не було. Були деякі невдалі рішення, через які ми були вимушені декілька разів змінювати архітектуру програми та переписувати деякі компоненти, але це не пов`язано з безпосередніми цілями лабораторної роботи.

З погляду теорії, не одразу зрозуміли, що собою являє стохастична функція, як вона має виглядати.

\section{Висновки.}

В процесі виконання цієї лабораторної роботи ми навчилися будувати оптимальні детерміністичну і стохастичну вирішуючі функції та обчислювати середні втрати і функцію втрат. Проаналізували обидві функції з огляду на значення середніх втрат і порівняли їх. Дійшли висновку, що обидві функції є оптимальними з погляду на очікувану помилку і з однаковою ефективністю дають найбільш ймовірний результат, хоча й не гарантовно правильний.

\section{Додаткові проміжні таблиці.}

\begin{table}[H]
    \centering
    \begin{adjustbox}{scale=0.64,center}
    \begin{tabular}{|*{21}{c|}}
    \hline
        $\mathbf{P(C)}$ & 0.0425 & 0.0475 & 0.0525 & 0.0500 & 0.0525 & 0.0450 & 0.0525 & 0.0550 & 0.0500 & 0.0450 & 0.0525 & 0.0500 & 0.0550 & 0.0500 & 0.0425 & 0.0550 & 0.0525 & 0.0525 & 0.0450 & 0.0525 \\ \hline
    \end{tabular}
\end{adjustbox}
    \caption{}
    \label{Tab6}
\end{table}

\begin{table}[H]
    \centering
    \begin{adjustbox}{scale=0.63,center}
    \begin{tabular}{|*{21}{c|}}
      \hline
        $\mathbf{P(M,C)}$ & $\mathbf{C_0}$ & $\mathbf{C_1}$ & $\mathbf{C_2}$ & $\mathbf{C_3}$ & $\mathbf{C_4}$ & $\mathbf{C_5}$ & $\mathbf{C_6}$ & $\mathbf{C_7}$ & $\mathbf{C_8}$ & $\mathbf{C_9}$ & $\mathbf{C_{10}}$ & $\mathbf{C_{11}}$ & $\mathbf{C_{12}}$ & $\mathbf{C_{13}}$ & $\mathbf{C_{14}}$ & $\mathbf{C_{15}}$ & $\mathbf{C_{16}}$ & $\mathbf{C_{17}}$ & $\mathbf{C_{18}}$ & $\mathbf{C_{19}}$ \\
    \hline
        $\mathbf{M_0}$ & 0.0000 & 0.0045 & 0.0045 & 0.0045 & 0.0000 & 0.0000 & 0.0000 & 0.0000 & 0.0045 & 0.0045 & 0.0000 & 0.0135 & 0.0135 & 0.0090 & 0.0000 & 0.0090 & 0.0090 & 0.0090 & 0.0000 & 0.0045 \\ \hline
        $\mathbf{M_1}$ & 0.0000 & 0.0000 & 0.0135 & 0.0090 & 0.0180 & 0.0045 & 0.0000 & 0.0045 & 0.0045 & 0.0045 & 0.0090 & 0.0045 & 0.0045 & 0.0000 & 0.0000 & 0.0045 & 0.0045 & 0.0000 & 0.0000 & 0.0045 \\ \hline
        $\mathbf{M_2}$ & 0.0045 & 0.0090 & 0.0000 & 0.0045 & 0.0000 & 0.0000 & 0.0180 & 0.0000 & 0.0045 & 0.0000 & 0.0090 & 0.0000 & 0.0045 & 0.0090 & 0.0045 & 0.0000 & 0.0045 & 0.0090 & 0.0045 & 0.0045 \\ \hline
        $\mathbf{M_3}$ & 0.0000 & 0.0000 & 0.0045 & 0.0000 & 0.0045 & 0.0045 & 0.0045 & 0.0225 & 0.0045 & 0.0000 & 0.0045 & 0.0000 & 0.0045 & 0.0000 & 0.0000 & 0.0135 & 0.0045 & 0.0045 & 0.0045 & 0.0090 \\ \hline
        $\mathbf{M_4}$ & 0.0040 & 0.0040 & 0.0020 & 0.0000 & 0.0000 & 0.0020 & 0.0040 & 0.0020 & 0.0020 & 0.0020 & 0.0020 & 0.0020 & 0.0040 & 0.0020 & 0.0000 & 0.0020 & 0.0000 & 0.0000 & 0.0060 & 0.0000 \\ \hline
        $\mathbf{M_5}$ & 0.0020 & 0.0000 & 0.0060 & 0.0000 & 0.0000 & 0.0000 & 0.0060 & 0.0000 & 0.0040 & 0.0020 & 0.0040 & 0.0020 & 0.0000 & 0.0020 & 0.0040 & 0.0020 & 0.0040 & 0.0000 & 0.0000 & 0.0020 \\ \hline
        $\mathbf{M_6}$ & 0.0040 & 0.0020 & 0.0020 & 0.0020 & 0.0000 & 0.0000 & 0.0000 & 0.0040 & 0.0020 & 0.0080 & 0.0020 & 0.0000 & 0.0000 & 0.0000 & 0.0040 & 0.0020 & 0.0020 & 0.0020 & 0.0020 & 0.0020 \\ \hline
        $\mathbf{M_7}$ & 0.0020 & 0.0060 & 0.0020 & 0.0040 & 0.0000 & 0.0000 & 0.0020 & 0.0000 & 0.0020 & 0.0020 & 0.0000 & 0.0020 & 0.0000 & 0.0040 & 0.0020 & 0.0080 & 0.0020 & 0.0000 & 0.0020 & 0.0000 \\ \hline
        $\mathbf{M_8}$ & 0.0000 & 0.0000 & 0.0020 & 0.0000 & 0.0060 & 0.0000 & 0.0020 & 0.0020 & 0.0000 & 0.0020 & 0.0040 & 0.0020 & 0.0020 & 0.0060 & 0.0020 & 0.0020 & 0.0000 & 0.0040 & 0.0020 & 0.0020 \\ \hline
        $\mathbf{M_9}$ & 0.0020 & 0.0000 & 0.0020 & 0.0040 & 0.0020 & 0.0020 & 0.0040 & 0.0000 & 0.0020 & 0.0000 & 0.0020 & 0.0000 & 0.0020 & 0.0040 & 0.0040 & 0.0000 & 0.0040 & 0.0040 & 0.0000 & 0.0020 \\ \hline
        $\mathbf{M_{10}}$ & 0.0000 & 0.0040 & 0.0000 & 0.0020 & 0.0040 & 0.0060 & 0.0020 & 0.0020 & 0.0020 & 0.0040 & 0.0000 & 0.0000 & 0.0020 & 0.0000 & 0.0020 & 0.0040 & 0.0020 & 0.0020 & 0.0000 & 0.0020 \\ \hline
        $\mathbf{M_{11}}$ & 0.0000 & 0.0060 & 0.0040 & 0.0040 & 0.0060 & 0.0000 & 0.0000 & 0.0020 & 0.0000 & 0.0020 & 0.0040 & 0.0000 & 0.0020 & 0.0000 & 0.0020 & 0.0000 & 0.0000 & 0.0020 & 0.0040 & 0.0020 \\ \hline
        $\mathbf{M_{12}}$ & 0.0020 & 0.0040 & 0.0000 & 0.0040 & 0.0040 & 0.0040 & 0.0000 & 0.0020 & 0.0020 & 0.0000 & 0.0020 & 0.0020 & 0.0000 & 0.0000 & 0.0000 & 0.0000 & 0.0020 & 0.0060 & 0.0000 & 0.0060 \\ \hline
        $\mathbf{M_{13}}$ & 0.0020 & 0.0020 & 0.0000 & 0.0000 & 0.0000 & 0.0060 & 0.0020 & 0.0040 & 0.0000 & 0.0000 & 0.0040 & 0.0060 & 0.0000 & 0.0000 & 0.0020 & 0.0040 & 0.0000 & 0.0000 & 0.0060 & 0.0020 \\ \hline
        $\mathbf{M_{14}}$ & 0.0040 & 0.0000 & 0.0020 & 0.0000 & 0.0020 & 0.0020 & 0.0020 & 0.0020 & 0.0000 & 0.0000 & 0.0020 & 0.0100 & 0.0040 & 0.0000 & 0.0000 & 0.0000 & 0.0060 & 0.0020 & 0.0020 & 0.0000 \\ \hline
        $\mathbf{M_{15}}$ & 0.0020 & 0.0020 & 0.0000 & 0.0040 & 0.0000 & 0.0020 & 0.0020 & 0.0020 & 0.0040 & 0.0000 & 0.0000 & 0.0020 & 0.0060 & 0.0020 & 0.0000 & 0.0000 & 0.0020 & 0.0040 & 0.0020 & 0.0040 \\ \hline
        $\mathbf{M_{16}}$ & 0.0040 & 0.0020 & 0.0000 & 0.0040 & 0.0040 & 0.0000 & 0.0020 & 0.0000 & 0.0040 & 0.0020 & 0.0040 & 0.0000 & 0.0000 & 0.0020 & 0.0060 & 0.0000 & 0.0000 & 0.0000 & 0.0040 & 0.0020 \\ \hline
        $\mathbf{M_{17}}$ & 0.0060 & 0.0000 & 0.0020 & 0.0020 & 0.0000 & 0.0020 & 0.0000 & 0.0020 & 0.0040 & 0.0040 & 0.0000 & 0.0020 & 0.0000 & 0.0060 & 0.0020 & 0.0000 & 0.0040 & 0.0000 & 0.0020 & 0.0020 \\ \hline
        $\mathbf{M_{18}}$ & 0.0020 & 0.0020 & 0.0040 & 0.0020 & 0.0020 & 0.0060 & 0.0020 & 0.0040 & 0.0020 & 0.0020 & 0.0000 & 0.0000 & 0.0020 & 0.0040 & 0.0000 & 0.0020 & 0.0020 & 0.0000 & 0.0000 & 0.0020 \\ \hline
        $\mathbf{M_{19}}$ & 0.0020 & 0.0000 & 0.0020 & 0.0000 & 0.0000 & 0.0040 & 0.0000 & 0.0000 & 0.0020 & 0.0060 & 0.0000 & 0.0020 & 0.0040 & 0.0000 & 0.0080 & 0.0020 & 0.0000 & 0.0040 & 0.0040 & 0.0000 \\ \hline
    \end{tabular}
    \end{adjustbox}
    \caption{}
    \label{Tab7}
\end{table}


\end{document}
